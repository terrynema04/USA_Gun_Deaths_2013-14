\documentclass[]{article}
\usepackage{lmodern}
\usepackage{amssymb,amsmath}
\usepackage{ifxetex,ifluatex}
\usepackage{fixltx2e} % provides \textsubscript
\ifnum 0\ifxetex 1\fi\ifluatex 1\fi=0 % if pdftex
  \usepackage[T1]{fontenc}
  \usepackage[utf8]{inputenc}
\else % if luatex or xelatex
  \ifxetex
    \usepackage{mathspec}
  \else
    \usepackage{fontspec}
  \fi
  \defaultfontfeatures{Ligatures=TeX,Scale=MatchLowercase}
\fi
% use upquote if available, for straight quotes in verbatim environments
\IfFileExists{upquote.sty}{\usepackage{upquote}}{}
% use microtype if available
\IfFileExists{microtype.sty}{%
\usepackage{microtype}
\UseMicrotypeSet[protrusion]{basicmath} % disable protrusion for tt fonts
}{}
\usepackage[margin=1in]{geometry}
\usepackage{hyperref}
\hypersetup{unicode=true,
            pdftitle={final\_Project},
            pdfauthor={Terrence Nemayire},
            pdfborder={0 0 0},
            breaklinks=true}
\urlstyle{same}  % don't use monospace font for urls
\usepackage{color}
\usepackage{fancyvrb}
\newcommand{\VerbBar}{|}
\newcommand{\VERB}{\Verb[commandchars=\\\{\}]}
\DefineVerbatimEnvironment{Highlighting}{Verbatim}{commandchars=\\\{\}}
% Add ',fontsize=\small' for more characters per line
\usepackage{framed}
\definecolor{shadecolor}{RGB}{248,248,248}
\newenvironment{Shaded}{\begin{snugshade}}{\end{snugshade}}
\newcommand{\KeywordTok}[1]{\textcolor[rgb]{0.13,0.29,0.53}{\textbf{#1}}}
\newcommand{\DataTypeTok}[1]{\textcolor[rgb]{0.13,0.29,0.53}{#1}}
\newcommand{\DecValTok}[1]{\textcolor[rgb]{0.00,0.00,0.81}{#1}}
\newcommand{\BaseNTok}[1]{\textcolor[rgb]{0.00,0.00,0.81}{#1}}
\newcommand{\FloatTok}[1]{\textcolor[rgb]{0.00,0.00,0.81}{#1}}
\newcommand{\ConstantTok}[1]{\textcolor[rgb]{0.00,0.00,0.00}{#1}}
\newcommand{\CharTok}[1]{\textcolor[rgb]{0.31,0.60,0.02}{#1}}
\newcommand{\SpecialCharTok}[1]{\textcolor[rgb]{0.00,0.00,0.00}{#1}}
\newcommand{\StringTok}[1]{\textcolor[rgb]{0.31,0.60,0.02}{#1}}
\newcommand{\VerbatimStringTok}[1]{\textcolor[rgb]{0.31,0.60,0.02}{#1}}
\newcommand{\SpecialStringTok}[1]{\textcolor[rgb]{0.31,0.60,0.02}{#1}}
\newcommand{\ImportTok}[1]{#1}
\newcommand{\CommentTok}[1]{\textcolor[rgb]{0.56,0.35,0.01}{\textit{#1}}}
\newcommand{\DocumentationTok}[1]{\textcolor[rgb]{0.56,0.35,0.01}{\textbf{\textit{#1}}}}
\newcommand{\AnnotationTok}[1]{\textcolor[rgb]{0.56,0.35,0.01}{\textbf{\textit{#1}}}}
\newcommand{\CommentVarTok}[1]{\textcolor[rgb]{0.56,0.35,0.01}{\textbf{\textit{#1}}}}
\newcommand{\OtherTok}[1]{\textcolor[rgb]{0.56,0.35,0.01}{#1}}
\newcommand{\FunctionTok}[1]{\textcolor[rgb]{0.00,0.00,0.00}{#1}}
\newcommand{\VariableTok}[1]{\textcolor[rgb]{0.00,0.00,0.00}{#1}}
\newcommand{\ControlFlowTok}[1]{\textcolor[rgb]{0.13,0.29,0.53}{\textbf{#1}}}
\newcommand{\OperatorTok}[1]{\textcolor[rgb]{0.81,0.36,0.00}{\textbf{#1}}}
\newcommand{\BuiltInTok}[1]{#1}
\newcommand{\ExtensionTok}[1]{#1}
\newcommand{\PreprocessorTok}[1]{\textcolor[rgb]{0.56,0.35,0.01}{\textit{#1}}}
\newcommand{\AttributeTok}[1]{\textcolor[rgb]{0.77,0.63,0.00}{#1}}
\newcommand{\RegionMarkerTok}[1]{#1}
\newcommand{\InformationTok}[1]{\textcolor[rgb]{0.56,0.35,0.01}{\textbf{\textit{#1}}}}
\newcommand{\WarningTok}[1]{\textcolor[rgb]{0.56,0.35,0.01}{\textbf{\textit{#1}}}}
\newcommand{\AlertTok}[1]{\textcolor[rgb]{0.94,0.16,0.16}{#1}}
\newcommand{\ErrorTok}[1]{\textcolor[rgb]{0.64,0.00,0.00}{\textbf{#1}}}
\newcommand{\NormalTok}[1]{#1}
\usepackage{graphicx,grffile}
\makeatletter
\def\maxwidth{\ifdim\Gin@nat@width>\linewidth\linewidth\else\Gin@nat@width\fi}
\def\maxheight{\ifdim\Gin@nat@height>\textheight\textheight\else\Gin@nat@height\fi}
\makeatother
% Scale images if necessary, so that they will not overflow the page
% margins by default, and it is still possible to overwrite the defaults
% using explicit options in \includegraphics[width, height, ...]{}
\setkeys{Gin}{width=\maxwidth,height=\maxheight,keepaspectratio}
\IfFileExists{parskip.sty}{%
\usepackage{parskip}
}{% else
\setlength{\parindent}{0pt}
\setlength{\parskip}{6pt plus 2pt minus 1pt}
}
\setlength{\emergencystretch}{3em}  % prevent overfull lines
\providecommand{\tightlist}{%
  \setlength{\itemsep}{0pt}\setlength{\parskip}{0pt}}
\setcounter{secnumdepth}{0}
% Redefines (sub)paragraphs to behave more like sections
\ifx\paragraph\undefined\else
\let\oldparagraph\paragraph
\renewcommand{\paragraph}[1]{\oldparagraph{#1}\mbox{}}
\fi
\ifx\subparagraph\undefined\else
\let\oldsubparagraph\subparagraph
\renewcommand{\subparagraph}[1]{\oldsubparagraph{#1}\mbox{}}
\fi

%%% Use protect on footnotes to avoid problems with footnotes in titles
\let\rmarkdownfootnote\footnote%
\def\footnote{\protect\rmarkdownfootnote}

%%% Change title format to be more compact
\usepackage{titling}

% Create subtitle command for use in maketitle
\providecommand{\subtitle}[1]{
  \posttitle{
    \begin{center}\large#1\end{center}
    }
}

\setlength{\droptitle}{-2em}

  \title{final\_Project}
    \pretitle{\vspace{\droptitle}\centering\huge}
  \posttitle{\par}
    \author{Terrence Nemayire}
    \preauthor{\centering\large\emph}
  \postauthor{\par}
      \predate{\centering\large\emph}
  \postdate{\par}
    \date{December 9, 2019}


\begin{document}
\maketitle

\begin{enumerate}
\def\labelenumi{\arabic{enumi}.}
\tightlist
\item
  Introduction
\end{enumerate}

This paper is on the recorded 2006 crimes committed in the various
cities of USA.Source data that will be used has been obtained from :
\url{https://data.world/ucr/crime-in-us-2006-offenses}. The dataset
being an excel file containing different variable of crim,e type. With
crimes patterns and population growth, models will be created that can
be used to predict or assists in decision making process. To embark on
this research various tools and softwares listed below will be needed to
help in answering the Research question

\begin{enumerate}
\def\labelenumi{\arabic{enumi}.}
\setcounter{enumi}{1}
\tightlist
\item
  Research Question:
\end{enumerate}

Does the size of population affect the type of crime or crimes committed
in particular towns in the United States of America? and the baseline
hypothesis is a) Population size affects type of crimes committed b) and
the null hpothesis being- Population size doesnt affect crimes commited

\begin{enumerate}
\def\labelenumi{\arabic{enumi}.}
\setcounter{enumi}{2}
\tightlist
\item
  Methodology and Tools
\end{enumerate}

To carry out the research analysis, the following tools will need to be
present and installed on a standard Computer withe the following minimu
requirements -Hardware ( 4GhZ cPU, 40g HDD, 4G RAM) -Minimum
Requirements to be installed -Windows Operating System 7/8/10 (32/64bit)
-Microsoft Office Package 2007 and above (Ms Excel and Ms Word) -R-
Studio Version 1.1.463 - \url{https://cran.rstudio.com/} -R version
3.6.1 -In R --Studio have update packages of the following; Knitr, Yaml,
Htmltools, caTools, Bitops, Rmarkdown, ggplot, ggplot2, LaTex -GitHub
account

4.1 Analysis -Loading data

The purpose of this analysis is to find out if there is a relationship
between population and specific crime types that occur in various
citiies. At this stage l calling the excel raw data file into R NB The
files was already cleaned from the original source and they will be no
need for a Key or data dictionary since all the variables are self
explanatory. Once the file has been loaded into a dataset in R. the
second stage will be to load the dataset into a dataframe that can be
manipulated easily by inbuilt libraries. The last part would be to view
the summary report of the dataframe for further analysis that will be
used in answering the Question. Code block below.

\begin{Shaded}
\begin{Highlighting}[]
\KeywordTok{library}\NormalTok{(readxl)}
\NormalTok{crimes <-}\StringTok{ }\KeywordTok{read_excel}\NormalTok{(}\StringTok{"C:/gun_violence_final/finals/project_data/crimefile.xlsx"}\NormalTok{)}
\NormalTok{crimeframe <-}\StringTok{ }\NormalTok{crimes [,}\KeywordTok{c}\NormalTok{(}\StringTok{"City"}\NormalTok{,}\StringTok{"population"}\NormalTok{,}\StringTok{"violent_crime"}\NormalTok{,}\StringTok{"murder"}\NormalTok{,}\StringTok{"rape"}\NormalTok{,}\StringTok{"robbery"}\NormalTok{,}\StringTok{"assault"}\NormalTok{)]}
\NormalTok{crimeframe}
\end{Highlighting}
\end{Shaded}

\begin{verbatim}
## # A tibble: 5,499 x 7
##    City       population violent_crime murder  rape robbery assault
##    <chr>           <dbl>         <dbl>  <dbl> <dbl>   <dbl>   <dbl>
##  1 Abbeville        2990            11      0     1       0      10
##  2 Adamsville       4889            44      0     2      13      29
##  3 Alabaster       27766            27      0     0       8      19
##  4 Aliceville       2487            33      1     1       2      29
##  5 Andalusia        8770            49      0     8       8      33
##  6 Anniston        23956           521     14    29     161     317
##  7 Ardmore          1116             1      0     0       0       1
##  8 Ashford          1949             3      0     0       0       3
##  9 Ashville         2451             5      0     1       1       3
## 10 Atmore           7598            69      0     2      11      56
## # ... with 5,489 more rows
\end{verbatim}

\begin{Shaded}
\begin{Highlighting}[]
\KeywordTok{summary}\NormalTok{ (crimeframe)}
\end{Highlighting}
\end{Shaded}

\begin{verbatim}
##      City             population      violent_crime         murder       
##  Length:5499        Min.   :     18   Min.   :    0.0   Min.   :  0.000  
##  Class :character   1st Qu.:   2474   1st Qu.:    3.0   1st Qu.:  0.000  
##  Mode  :character   Median :   6500   Median :   13.0   Median :  0.000  
##                     Mean   :  24266   Mean   :  136.9   Mean   :  1.749  
##                     3rd Qu.:  18230   3rd Qu.:   51.0   3rd Qu.:  0.000  
##                     Max.   :8165001   Max.   :52086.0   Max.   :596.000  
##                     NA's   :1         NA's   :2                          
##       rape             robbery            assault        
##  Min.   :   0.000   Min.   :    0.00   Min.   :    0.00  
##  1st Qu.:   0.000   1st Qu.:    0.00   1st Qu.:    2.00  
##  Median :   1.000   Median :    2.00   Median :    9.00  
##  Mean   :   7.931   Mean   :   51.38   Mean   :   82.56  
##  3rd Qu.:   5.000   3rd Qu.:   11.00   3rd Qu.:   33.00  
##  Max.   :1071.000   Max.   :23511.00   Max.   :26908.00  
##  NA's   :1                             NA's   :2
\end{verbatim}

4.2 Analysis - Finding coerrelation between the variables

After analysis of the dataframe, the various coefficients of the
variables, alot had a result of NA as a coefficient. These results of NA
indicates that the variables in question are not linearly related to the
other variables, or the data from the dataset is noty sufficient enough
to prove a meaning to the level of signficance and for multiple
regression, the NA coeeficients depicts that the variables do not add
much value to the models or affects the response variable (Y). to find
coefficient, l have to create anew dataframe colled corcrimeframe
eliminating column on cities.

\begin{Shaded}
\begin{Highlighting}[]
\NormalTok{corcrimeframe <-}\StringTok{ }\NormalTok{crimeframe [,}\KeywordTok{c}\NormalTok{(}\StringTok{"population"}\NormalTok{,}\StringTok{"violent_crime"}\NormalTok{,}\StringTok{"murder"}\NormalTok{,}\StringTok{"rape"}\NormalTok{,}\StringTok{"robbery"}\NormalTok{,}\StringTok{"assault"}\NormalTok{)]}
\KeywordTok{cor}\NormalTok{(corcrimeframe)}
\end{Highlighting}
\end{Shaded}

\begin{verbatim}
##               population violent_crime   murder rape  robbery assault
## population             1            NA       NA   NA       NA      NA
## violent_crime         NA             1       NA   NA       NA      NA
## murder                NA            NA 1.000000   NA 0.827243      NA
## rape                  NA            NA       NA    1       NA      NA
## robbery               NA            NA 0.827243   NA 1.000000      NA
## assault               NA            NA       NA   NA       NA       1
\end{verbatim}

4.3 Anlysis - Linear Regression Calculatioin

Linear regression in this case is used to establish a linear
relationship (a mathematical formula) between the predictor variable(s)
and the response variable, so that, l can use this formula to estimate
the value of the response Y, when only the predictors (Xs) values are
known. the Resultant formulae will be in the form of Y = AX + B Now l
will calculate the Linear regression model from these variables,

NB: The response variable (Y) is murder cases reported and the
population is the predictor variable (X) murder =Intercept + (beta *
population)

\begin{Shaded}
\begin{Highlighting}[]
\NormalTok{linearMod <-}\StringTok{ }\KeywordTok{lm}\NormalTok{(murder }\OperatorTok{~}\StringTok{ }\NormalTok{population, }\DataTypeTok{data=}\NormalTok{corcrimeframe)}
\KeywordTok{print}\NormalTok{(linearMod)}
\end{Highlighting}
\end{Shaded}

\begin{verbatim}
## 
## Call:
## lm(formula = murder ~ population, data = corcrimeframe)
## 
## Coefficients:
## (Intercept)   population  
##  -3.250e-01    8.547e-05
\end{verbatim}

\begin{Shaded}
\begin{Highlighting}[]
\KeywordTok{summary}\NormalTok{ (linearMod)}
\end{Highlighting}
\end{Shaded}

\begin{verbatim}
## 
## Call:
## lm(formula = murder ~ population, data = corcrimeframe)
## 
## Residuals:
##     Min      1Q  Median      3Q     Max 
## -240.93   -0.68   -0.04    0.20  342.73 
## 
## Coefficients:
##               Estimate Std. Error t value Pr(>|t|)    
## (Intercept) -3.250e-01  1.262e-01  -2.576     0.01 *  
## population   8.547e-05  8.765e-07  97.510   <2e-16 ***
## ---
## Signif. codes:  0 '***' 0.001 '**' 0.01 '*' 0.05 '.' 0.1 ' ' 1
## 
## Residual standard error: 9.221 on 5496 degrees of freedom
##   (1 observation deleted due to missingness)
## Multiple R-squared:  0.6337, Adjusted R-squared:  0.6336 
## F-statistic:  9508 on 1 and 5496 DF,  p-value: < 2.2e-16
\end{verbatim}

The linear regression model for the two variables (murder and
population) is as follow murder =-3.250e-01 + 8.547e-05(population)
hence the mathematical model is: Y = -3.250e-01 + 8.547e-05X (Hence with
this model l can try to predict the possible number of murders if
population in a city increases or decreases), since from the coefficents
table, a strong signficant relationship exists. Using the inbuilt
summary function in R, l can find the p values of the model to determine
model's significance statistically. The p value statistical signficance
of 2.2e-16 is above the pre-determined significance level of 0.05 hence
the population variable has more signficance in this linear regression
model, hence to say the more the population the more the number of
recorded murder cases and the model can be used accurately to predict.

Below is a scatter plot diagram based on the Pearson Coeefficient Model.
The product-moment correlation coefficient is a measure of the strength
of the linear relationship between the two variables (Population and
Murder)

\begin{Shaded}
\begin{Highlighting}[]
\KeywordTok{library}\NormalTok{(}\StringTok{"ggpubr"}\NormalTok{)}
\end{Highlighting}
\end{Shaded}

\begin{verbatim}
## Loading required package: ggplot2
\end{verbatim}

\begin{verbatim}
## Loading required package: magrittr
\end{verbatim}

\begin{Shaded}
\begin{Highlighting}[]
\KeywordTok{ggscatter}\NormalTok{(corcrimeframe, }\DataTypeTok{x =} \StringTok{"population"}\NormalTok{, }\DataTypeTok{y =} \StringTok{"murder"}\NormalTok{, }
          \DataTypeTok{add =} \StringTok{"reg.line"}\NormalTok{, }\DataTypeTok{conf.int =} \OtherTok{TRUE}\NormalTok{, }
          \DataTypeTok{cor.coef =} \OtherTok{TRUE}\NormalTok{, }\DataTypeTok{cor.method =} \StringTok{"pearson"}\NormalTok{)}
\end{Highlighting}
\end{Shaded}

\begin{verbatim}
## Warning: Removed 1 rows containing non-finite values (stat_smooth).
\end{verbatim}

\begin{verbatim}
## Warning: Removed 1 rows containing non-finite values (stat_cor).
\end{verbatim}

\begin{verbatim}
## Warning: Removed 1 rows containing missing values (geom_point).
\end{verbatim}

\includegraphics{gun_violence_analysis_files/figure-latex/unnamed-chunk-4-1.pdf}

\begin{Shaded}
\begin{Highlighting}[]
\NormalTok{corgraph <-}\StringTok{ }\KeywordTok{cor.test}\NormalTok{(corcrimeframe}\OperatorTok{$}\NormalTok{population, corcrimeframe}\OperatorTok{$}\NormalTok{murder) }
\NormalTok{                    method =}\StringTok{ }\NormalTok{(}\StringTok{"pearson"}\NormalTok{)}
\NormalTok{corgraph}
\end{Highlighting}
\end{Shaded}

\begin{verbatim}
## 
##  Pearson's product-moment correlation
## 
## data:  corcrimeframe$population and corcrimeframe$murder
## t = 97.51, df = 5496, p-value < 2.2e-16
## alternative hypothesis: true correlation is not equal to 0
## 95 percent confidence interval:
##  0.7861630 0.8055371
## sample estimates:
##       cor 
## 0.7960539
\end{verbatim}

4.4 Anlysis -Multiple Linear Regression

Multiple linear Regression, is a statistical technique that uses several
variables to predict the outcome of a response variable. The goal of
multiple linear regression (MLR) is to model the linear relationship
between the independent variables and response-dependent variable. From
the question above, I intend to answer the question, Is there a relation
between the number of murder case against changes from other variables
(population, assult cases, rape,robbery and violent crimes). The
expected regression model that best suits is one which is proven
statistically to have a high degree of freedom value, stigma error rate,
p value using the t test and the Adjusted R-Squared value. The model
equation will be as y = a + b1x1 + b2x2 +\ldots{}bnxn, where x1, x2..xn
repreesnt the predictor variable, b1, b2..bn being the variable
coefficients and the value of a being the intercept (constant)

\begin{enumerate}
\def\labelenumi{\alph{enumi})}
\item
  MultiModel\_1 -building the first multi regression model using all the
  variables, Y is murder and X predictor variants being other 6
  variables, (population, violent\_crime, murder, rape, robbery
  ,assault)
\item
  MultiModel\_2- building the second multi regression model
\end{enumerate}

\begin{Shaded}
\begin{Highlighting}[]
\NormalTok{MultiModel_}\DecValTok{1}\NormalTok{ <-}\StringTok{ }\KeywordTok{lm}\NormalTok{(murder }\OperatorTok{~}\StringTok{ }\NormalTok{population}\OperatorTok{+}\NormalTok{violent_crime}\OperatorTok{+}\NormalTok{murder}\OperatorTok{+}\NormalTok{rape}\OperatorTok{+}\NormalTok{robbery}\OperatorTok{+}\NormalTok{assault , }\DataTypeTok{data=}\NormalTok{corcrimeframe)}
\end{Highlighting}
\end{Shaded}

\begin{verbatim}
## Warning in model.matrix.default(mt, mf, contrasts): the response appeared
## on the right-hand side and was dropped
\end{verbatim}

\begin{verbatim}
## Warning in model.matrix.default(mt, mf, contrasts): problem with term 3 in
## model.matrix: no columns are assigned
\end{verbatim}

\begin{Shaded}
\begin{Highlighting}[]
\KeywordTok{summary}\NormalTok{ (MultiModel_}\DecValTok{1}\NormalTok{)}
\end{Highlighting}
\end{Shaded}

\begin{verbatim}
## 
## Call:
## lm(formula = murder ~ population + violent_crime + murder + rape + 
##     robbery + assault, data = corcrimeframe)
## 
## Residuals:
##     Min      1Q  Median      3Q     Max 
## -60.686  -0.324  -0.206   0.135 231.820 
## 
## Coefficients:
##                 Estimate Std. Error t value Pr(>|t|)    
## (Intercept)    2.922e-01  7.300e-02   4.003 6.34e-05 ***
## population    -5.310e-05  1.609e-06 -33.009  < 2e-16 ***
## violent_crime  1.616e-02  2.198e-04  73.527  < 2e-16 ***
## rape           7.540e-03  4.030e-03   1.871   0.0614 .  
## robbery        1.038e-02  6.702e-04  15.493  < 2e-16 ***
## assault       -7.410e-04  5.309e-04  -1.396   0.1628    
## ---
## Signif. codes:  0 '***' 0.001 '**' 0.01 '*' 0.05 '.' 0.1 ' ' 1
## 
## Residual standard error: 5.096 on 5489 degrees of freedom
##   (4 observations deleted due to missingness)
## Multiple R-squared:  0.8883, Adjusted R-squared:  0.8882 
## F-statistic:  8727 on 5 and 5489 DF,  p-value: < 2.2e-16
\end{verbatim}

\begin{Shaded}
\begin{Highlighting}[]
\NormalTok{MultiModel_}\DecValTok{2}\NormalTok{ <-}\StringTok{ }\KeywordTok{lm}\NormalTok{(murder }\OperatorTok{~}\StringTok{ }\NormalTok{population}\OperatorTok{+}\NormalTok{violent_crime}\OperatorTok{+}\NormalTok{murder}\OperatorTok{+}\NormalTok{robbery, }\DataTypeTok{data=}\NormalTok{corcrimeframe)  }\CommentTok{#}
\end{Highlighting}
\end{Shaded}

\begin{verbatim}
## Warning in model.matrix.default(mt, mf, contrasts): the response appeared
## on the right-hand side and was dropped

## Warning in model.matrix.default(mt, mf, contrasts): problem with term 3 in
## model.matrix: no columns are assigned
\end{verbatim}

\begin{Shaded}
\begin{Highlighting}[]
\KeywordTok{summary}\NormalTok{ (MultiModel_}\DecValTok{2}\NormalTok{)}
\end{Highlighting}
\end{Shaded}

\begin{verbatim}
## 
## Call:
## lm(formula = murder ~ population + violent_crime + murder + robbery, 
##     data = corcrimeframe)
## 
## Residuals:
##     Min      1Q  Median      3Q     Max 
## -56.998  -0.337  -0.218   0.130 231.860 
## 
## Coefficients:
##                 Estimate Std. Error t value Pr(>|t|)    
## (Intercept)    3.090e-01  7.072e-02   4.369 1.27e-05 ***
## population    -5.310e-05  1.597e-06 -33.248  < 2e-16 ***
## violent_crime  1.629e-02  1.640e-04  99.369  < 2e-16 ***
## robbery        9.704e-03  4.284e-04  22.649  < 2e-16 ***
## ---
## Signif. codes:  0 '***' 0.001 '**' 0.01 '*' 0.05 '.' 0.1 ' ' 1
## 
## Residual standard error: 5.098 on 5492 degrees of freedom
##   (3 observations deleted due to missingness)
## Multiple R-squared:  0.8881, Adjusted R-squared:  0.8881 
## F-statistic: 1.453e+04 on 3 and 5492 DF,  p-value: < 2.2e-16
\end{verbatim}

\begin{Shaded}
\begin{Highlighting}[]
\NormalTok{## .}
\end{Highlighting}
\end{Shaded}

5 Results

\begin{enumerate}
\def\labelenumi{\alph{enumi})}
\item
  Model\_2 Y= 0.29 + (-0.00)population + (0.016)violent\_crime +
  (0.01)rape + (0.01)robbery + (-0,00)assault From the first model,
  predictor variables (Rape and Assault and negligable and has no
  signficance in the model that can answer the resarch question, hence
  the second model l will remove the 2 negligable variables and improve
  model.
\item
  Model\_2 Y= 0.31 + (-0.00)population + (0.016)violent\_crime
  +(0.01)robbery From the second model, all the variables have a p value
  that is signaficanT based on the t-test and the variables have a high
  signficant effect to the response variable. The Adjusted R-squared
  test IN the new model has a 89\% accuracy rate for the model using the
  (population, violent\_crime and robbery variables) hence l can safely
  conclude to say Model 2 can statistically be used for further analysis
  or prediction.
\end{enumerate}

\begin{Shaded}
\begin{Highlighting}[]
\KeywordTok{print}\NormalTok{(MultiModel_}\DecValTok{1}\NormalTok{)}
\end{Highlighting}
\end{Shaded}

\begin{verbatim}
## 
## Call:
## lm(formula = murder ~ population + violent_crime + murder + rape + 
##     robbery + assault, data = corcrimeframe)
## 
## Coefficients:
##   (Intercept)     population  violent_crime           rape        robbery  
##     0.2921880     -0.0000531      0.0161600      0.0075399      0.0103834  
##       assault  
##    -0.0007410
\end{verbatim}

\begin{Shaded}
\begin{Highlighting}[]
\KeywordTok{print}\NormalTok{(MultiModel_}\DecValTok{2}\NormalTok{)}
\end{Highlighting}
\end{Shaded}

\begin{verbatim}
## 
## Call:
## lm(formula = murder ~ population + violent_crime + murder + robbery, 
##     data = corcrimeframe)
## 
## Coefficients:
##   (Intercept)     population  violent_crime        robbery  
##     0.3089815     -0.0000531      0.0162939      0.0097039
\end{verbatim}

\begin{Shaded}
\begin{Highlighting}[]
\KeywordTok{library}\NormalTok{(scatterplot3d)}
\KeywordTok{attach}\NormalTok{(crimes)}
\KeywordTok{scatterplot3d}\NormalTok{(population,murder,robbery,violent_crime,}\DataTypeTok{main =}\StringTok{"Crime Statistis main 3D scatterplot"}\NormalTok{)}
\end{Highlighting}
\end{Shaded}

\includegraphics{gun_violence_analysis_files/figure-latex/unnamed-chunk-6-1.pdf}

6 Conclusion

The experiment has been conducted successfully to establish the strength
of relationships between the various variables in the dataset.
Regression models have been formulated for future use in prediction and
aid in decison making process in crime analysis. From the dataset used
for this experiment, l can statistically say that murder crimes are
related to the size of the population in a city and also other variables
have little of low signficance levels in determing murder crimes.


\end{document}
